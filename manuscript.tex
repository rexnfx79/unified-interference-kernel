\documentclass[12pt,a4paper]{article}
\usepackage[utf8]{inputenc}
\usepackage[T1]{fontenc}
\usepackage{amsmath,amssymb,amsthm}
\usepackage{graphicx}
\usepackage{subcaption}
\usepackage{hyperref}
\usepackage{natbib}
\usepackage{booktabs}
\usepackage{siunitx}
\usepackage{geometry}
\geometry{margin=1in}

% Custom commands
\newcommand{\abs}[1]{\left|#1\right|}
\newcommand{\expval}[1]{\langle#1\rangle}
\newcommand{\vect}[1]{\boldsymbol{#1}}

% Title and author
\title{Universal Interference Kernel: Three Projection Regimes for Fermion Flavor}
\author{Alexander Seto}
\date{January 18, 2026}

\begin{document}

\maketitle

\begin{abstract}
We present a unified framework for organizing Yukawa couplings through envelope suppression and phase interference. The model employs a single universal kernel form across all fermion sectors, with different sectors corresponding to distinct projection regimes of the same underlying structure. Three regimes are identified: envelope-dominated (quarks), phase-sensitive (charged leptons), and metric-dominated (neutrinos). We demonstrate that this framework successfully reproduces CKM mixing angles, charged lepton masses, and PMNS mixing angles with high precision. Key findings include universal parameter values across sectors (envelope width $\sigma \approx 4$--5, phase parameter $\alpha \approx 2.5$--3.0), exponential decay patterns in Pareto boundaries, and the emergence of neutrino anarchy from information loss under metric compression. Survivor analysis reveals 60\% success rate for charged leptons and 45\% for neutrinos, with all survivor Z-scores below 1.0, demonstrating excellent statistical agreement with experimental data.
\end{abstract}

\noindent\textbf{Keywords:} flavor physics, Yukawa couplings, CKM matrix, PMNS matrix, split fermions, neutrino anarchy

\section{Introduction}

The flavor puzzle in the Standard Model---the origin of fermion masses and mixing angles---remains one of the most fundamental open questions in particle physics. Despite decades of theoretical effort, no compelling explanation has emerged for why the quark and lepton mixing matrices exhibit their observed hierarchical structures.

The split-fermion framework \citep{ArkaniHamed2000} provided an elegant geometric approach, positing that fermions are localized at different positions in an extra dimension, with Yukawa couplings determined by wavefunction overlaps. However, this approach alone cannot account for the full complexity of observed mixing patterns, particularly in the neutrino sector where mixing angles are large.

We propose that a universal interference kernel, combining envelope suppression with phase interference, can organize all Yukawa couplings across fermion sectors. The key insight is that different sectors probe different \textit{projection regimes} of the same underlying kernel: quarks probe the envelope-dominated regime, charged leptons probe the phase-sensitive regime, and neutrinos probe the metric-dominated regime.

This work presents comprehensive optimization results across all three fermion sectors, demonstrating that the three-regime framework successfully reproduces experimental observables with high precision. We identify universal parameter values across sectors, characterize Pareto boundary shapes, and provide evidence for the emergence of neutrino anarchy from information loss under metric compression.

The paper is organized as follows: Section 2 introduces the universal interference kernel and three projection regimes. Section 3 describes the optimization methodology. Sections 4--5 present results for quarks and leptons, respectively. Section 6 discusses interpretations and implications. Section 7 concludes.

\section{Model}

\subsection{Universal Interference Kernel}

The universal interference kernel organizes Yukawa couplings through a combination of envelope suppression and phase interference:

\begin{equation}
Y_{ij} = \exp\left(-\frac{d_{ij}^2}{2\sigma^2}\right) \times \left[1 + \varepsilon \exp(i\Phi_{ij})\right],
\end{equation}

where $d_{ij} = |x_i - x_j|$ is the distance in an internal flavor coordinate (not spacetime), and $\Phi_{ij}$ is a phase structure:

\begin{equation}
\Phi_{ij} = \alpha + k\frac{x_i + x_j}{2} + \eta(x_i - x_j).
\end{equation}

The parameters are:
\begin{itemize}
\item $\sigma$: envelope width, controlling the range of overlap
\item $\varepsilon$: interference strength, controlling the amplitude of phase-dependent modulation
\item $\alpha, k, \eta$: phase parameters determining interference patterns
\item $x_i, x_j$: discrete coordinates in flavor space
\end{itemize}

The envelope term $\exp(-d_{ij}^2/(2\sigma^2))$ provides hierarchical suppression with distance, while the interference term $1 + \varepsilon \exp(i\Phi_{ij})$ introduces phase-dependent modulations that can enhance or suppress couplings.

\subsection{Discrete Geometry}

Fermions are assigned discrete coordinates in a three-dimensional flavor space. This discrete geometry framework provides a minimal parameterization of flavor structure, with each coordinate assignment representing a distinct geometry.

For quarks, we have:
\begin{itemize}
\item Left-handed doublets: $(Q_1, Q_2, 0)$ where $Q_1, Q_2$ are integers
\item Up-type right-handed singlets: $(U_1, U_2, U_3)$ where $U_1, U_2, U_3$ are integers
\item Down-type right-handed singlets: $(D_1, D_2, D_3)$ where $D_1, D_2, D_3$ are integers
\end{itemize}

For charged leptons:
\begin{itemize}
\item Left-handed doublets: $(L_1, L_2, 0)$ where $L_1, L_2$ are integers
\item Right-handed singlets: $(E_1, E_2, E_3)$ where $E_1, E_2, E_3$ are integers
\end{itemize}

For neutrinos:
\begin{itemize}
\item Left-handed doublets: $(L_1, L_2, 0)$ (same as charged leptons)
\item Right-handed neutrinos: $(N_1, N_2, N_3)$ where $N_1, N_2, N_3$ are integers
\item Right-handed charged leptons: $(E_1, E_2, E_3)$ (affects PMNS via SVD)
\end{itemize}

The Yukawa matrix is constructed by evaluating the kernel between all pairs of left-handed and right-handed fermion coordinates. For example, for quarks, the up-type Yukawa matrix has elements:
\begin{equation}
Y_{u,ij} = K(Q_i, U_j; \sigma, k, \alpha, \eta, \varepsilon_u),
\end{equation}
where $K$ is the kernel function, and similarly for down-type quarks with coordinates $D_j$ and parameter $\varepsilon_d$.

The resulting $3 \times 3$ complex matrix is then diagonalized via singular value decomposition (SVD) to extract physical observables. For quark and lepton sectors, SVD yields:
\begin{equation}
Y = U \Sigma V^\dagger,
\end{equation}
where $U$ and $V$ are unitary matrices encoding mixing angles, and $\Sigma$ is a diagonal matrix encoding masses. For neutrinos, both the neutrino Yukawa matrix $Y_\nu$ and charged lepton Yukawa matrix $Y_e$ are diagonalized, and the PMNS mixing matrix is extracted as $U_{\text{PMNS}} = U_e^\dagger U_\nu$ where $U_e$ and $U_\nu$ come from SVD of $Y_e$ and $Y_\nu$ respectively.

\subsection{Three Projection Regimes}

The key insight of this work is that different fermion sectors correspond to different projection regimes of the same universal kernel:

\textbf{Envelope-Dominated Regime (Quarks):} The quark sector uses baseline kernel parameters $(\sigma, k, \alpha, \eta)$ with sector-specific interference strengths $(\varepsilon_u, \varepsilon_d)$ for up- and down-type quarks. The envelope suppression dominates the hierarchy, with phase interference providing subdominant corrections. This regime is optimized to reproduce CKM mixing angles and quark masses.

\textbf{Phase-Sensitive Regime (Charged Leptons):} The charged lepton sector uses variable phase parameters $(k_e, \eta_e)$ distinct from the quark baseline, while maintaining similar envelope parameters. Phase interference becomes more important, allowing for resolution of the muon mass hierarchy. This regime targets charged lepton masses with high precision.

\textbf{Metric-Dominated Regime (Neutrinos):} The neutrino sector employs envelope compression via a compression factor $g_{\text{env}}$, where the effective envelope width becomes $\sigma_{\text{eff}} = \sigma \times g_{\text{env}}$. This compression leads to information loss, resulting in emergent anarchy in PMNS mixing angles. Additionally, charged lepton Yukawa matrices use lepton-specific phase parameters $(k_e, \eta_e, \varepsilon_e)$, while neutrino Yukawa matrices use neutrino-specific parameters $(k, \eta, \varepsilon_\nu)$.

\subsection{Observables}

Physical observables are extracted from Yukawa matrices via singular value decomposition (SVD). The extraction procedure depends on the fermion sector:

\textbf{Quark Sector:} From the up- and down-type Yukawa matrices $Y_u$ and $Y_d$, we perform SVD:
\begin{equation}
Y_u = U_u \Sigma_u V_u^\dagger, \quad Y_d = U_d \Sigma_d V_d^\dagger,
\end{equation}
where $\Sigma_u$ and $\Sigma_d$ are diagonal matrices containing masses, and $U_u, V_u, U_d, V_d$ are unitary matrices. The CKM mixing matrix is:
\begin{equation}
V_{\text{CKM}} = U_u^\dagger U_d,
\end{equation}
from which we extract the mixing angles and CP-violating phase. We focus on the elements $V_{us}, V_{cb}, V_{ub}$ which are precisely measured.

Quark masses are extracted from $\Sigma_u$ and $\Sigma_d$. We compare to PDG 2024 values \citep{PDG2024}:
\begin{itemize}
\item Up-type: $m_u, m_c$ (charm quark mass $m_c = 1.27$ GeV is most important)
\item Down-type: $m_d, m_s, m_b$
\end{itemize}

\textbf{Charged Lepton Sector:} From the charged lepton Yukawa matrix $Y_e$, we perform SVD:
\begin{equation}
Y_e = U_e \Sigma_e V_e^\dagger,
\end{equation}
where $\Sigma_e$ contains the charged lepton masses. We extract $m_e, m_\mu, m_\tau$ and compare to PDG 2024 values: $m_e = 0.000511$ GeV, $m_\mu = 0.105658$ GeV, $m_\tau = 1.77686$ GeV.

\textbf{Neutrino Sector:} Both neutrino and charged lepton Yukawa matrices are diagonalized:
\begin{equation}
Y_\nu = U_\nu \Sigma_\nu V_\nu^\dagger, \quad Y_e = U_e \Sigma_e V_e^\dagger.
\end{equation}
The PMNS mixing matrix is:
\begin{equation}
U_{\text{PMNS}} = U_e^\dagger U_\nu,
\end{equation}
from which we extract mixing angles $\theta_{12}, \theta_{23}, \theta_{13}$ using the standard parametrization. We also extract neutrino masses from $\Sigma_\nu$ and compute mass-squared differences $\Delta m_{21}^2, \Delta m_{31}^2$.

\textbf{Loss Functions:} We define loss functions to compare predictions to experimental values. For mixing angles, we use relative squared errors:
\begin{equation}
L_{\text{mixing}} = \sum_{i} \left(\frac{O_i^{\text{pred}} - O_i^{\text{exp}}}{O_i^{\text{exp}}}\right)^2,
\end{equation}
where $O_i$ are mixing matrix elements or angles. For masses, we use squared log-ratio errors:
\begin{equation}
L_{\text{mass}} = \sum_{i} \left(\log\frac{m_i^{\text{pred}}}{m_i^{\text{exp}}}\right)^2,
\end{equation}
which is sensitive to relative errors and appropriate for hierarchical masses spanning many orders of magnitude.

\subsection{Scope and Epistemological Framework}

This model operates at the effective field theory level, providing a structural framework for organizing Yukawa couplings rather than a fundamental theory. The claims are:
\begin{enumerate}
\item A single universal kernel form can organize all Yukawa couplings
\item Different fermion sectors correspond to different sampling regimes
\item Neutrino anarchy emerges from metric-dominated projection (information loss)
\end{enumerate}

These are structural observations within this model class, not claims about fundamental physics. The model provides a framework for understanding flavor patterns, but makes no claims about UV completion or underlying microscopic theory.

\section{Methods}

\subsection{Optimization Strategy}

We employ differential evolution \citep{Storn1997} to optimize kernel parameters for each geometry (discrete coordinate assignment). For each sector, we:
\begin{enumerate}
\item Generate discrete geometries systematically
\item For each geometry, optimize continuous parameters $(\sigma, k, \alpha, \eta, \varepsilon)$
\item Use multiple random seeds (typically 5) to ensure robustness
\item Select best result across seeds for each geometry
\end{enumerate}

The optimization minimizes a combined loss function:
\begin{equation}
L_{\text{total}} = L_{\text{mass}} + \lambda L_{\text{mixing}},
\end{equation}
where $L_{\text{mass}}$ is the sum of squared log-ratio errors for masses, $L_{\text{mixing}}$ is the sum of relative squared errors for mixing angles, and $\lambda$ is a weighting factor (typically 5.0) to balance the two objectives.

We use the following hyperparameters:
\begin{itemize}
\item Maximum iterations: 200 per optimization
\item Random seeds: 5 per geometry
\item Population size: 15 $\times$ number of parameters (differential evolution default)
\item Tolerance: $10^{-6}$ for convergence
\end{itemize}

\subsection{Sector-Specific Methods}

\subsubsection{Quark Sector Optimization}

We optimize over discrete geometries with $Q, U, D$ coordinates up to maximum coordinate value 5, generating 1000 geometries. The discrete coordinates are systematically enumerated, with each geometry representing a unique assignment of flavor space positions to the three generations of quarks.

For each geometry, we optimize six continuous parameters: $(\sigma, k, \alpha, \eta, \varepsilon_u, \varepsilon_d)$. The parameter bounds are chosen based on prior exploratory analysis:
\begin{itemize}
\item $\sigma \in [0.5, 6.0]$: Envelope width, controlling the range of overlap
\item $k \in [0.1, 2.0]$: Phase parameter controlling symmetric part of phase
\item $\alpha \in [0, 2\pi]$: Global phase shift
\item $\eta \in [1.0, 5.0]$: Phase parameter controlling antisymmetric part
\item $\varepsilon_u, \varepsilon_d \in [0.01, 0.5]$: Interference strengths for up- and down-type quarks
\end{itemize}

The objective function combines CKM mixing loss and mass loss:
\begin{equation}
L_{\text{quark}} = L_{\text{mass}} + 5.0 \times L_{\text{CKM}} + L_{\text{penalty}},
\end{equation}
where $L_{\text{mass}}$ is the sum of squared log-ratio errors for quark masses, $L_{\text{CKM}}$ is the sum of relative squared errors for CKM matrix elements $V_{us}, V_{cb}, V_{ub}$, and $L_{\text{penalty}}$ includes penalties for masses falling outside acceptable ranges.

Each geometry is optimized with 5 different random seeds to ensure robustness against local minima, and the best result across seeds is retained.

\subsubsection{Charged Lepton Sector Optimization}

We optimize over 100 geometries with $L, E$ coordinates, where $L = (L_1, L_2, 0)$ represents left-handed lepton doublets and $E = (E_1, E_2, E_3)$ represents right-handed lepton singlets.

The phase-sensitive regime uses five parameters: $(\sigma, k_e, \alpha, \eta_e, \varepsilon_e)$ where $k_e$ and $\eta_e$ are phase parameters specific to leptons, allowing variation from the quark baseline. This phase variation is crucial for resolving the muon mass hierarchy, as the phase-sensitive regime enables fine-tuning of interference patterns.

The objective function focuses on charged lepton masses:
\begin{equation}
L_{\text{lepton}} = L_{\text{mass}} + L_{\text{penalty}},
\end{equation}
where $L_{\text{mass}}$ includes errors for $m_e, m_\mu, m_\tau$, and penalties ensure masses fall within acceptable ranges. The phase-sensitive regime achieves exceptional precision, with 60\% of geometries matching all experimental mass ranges simultaneously.

Each optimization uses 5 random seeds with 200 iterations per seed, with the best result retained.

\subsubsection{Neutrino Sector Optimization}

We optimize over 96 geometries $\times$ 5 values of $g_{\text{env}} \in [0.5, 0.6, 0.65, 0.7]$ (5 discrete steps), totaling 480 optimizations. The envelope compression factor $g_{\text{env}}$ controls the effective envelope width $\sigma_{\text{eff}} = \sigma \times g_{\text{env}}$.

The metric-dominated regime requires nine parameters: $(\sigma, k, \alpha, \eta, \varepsilon_\nu, k_e, \eta_e, \varepsilon_e, g_{\text{env}})$ where:
\begin{itemize}
\item $k, \eta, \varepsilon_\nu$ control the neutrino Yukawa matrix $Y_\nu$
\item $k_e, \eta_e, \varepsilon_e$ control the charged lepton Yukawa matrix $Y_e$ (which affects PMNS extraction via SVD)
\item $g_{\text{env}}$ controls envelope compression for the neutrino sector
\end{itemize}

The objective function combines PMNS mixing loss and neutrino mass loss:
\begin{equation}
L_{\text{neutrino}} = L_{\text{mass}} + 5.0 \times L_{\text{PMNS}} + L_{\text{m1,penalty}},
\end{equation}
where $L_{\text{PMNS}}$ includes errors for $\theta_{12}, \theta_{23}, \theta_{13}$, and $L_{\text{m1,penalty}}$ ensures the lightest neutrino mass falls within acceptable bounds.

The compression factor $g_{\text{env}} < 1$ leads to information loss, which manifests as emergent anarchy in PMNS mixing angles. Optimal compression occurs at $g_{\text{env}} \approx 0.5$--0.6, where 45\% of geometries match experimental PMNS angles.

\section{Results: Quark Sector}

\subsection{Robust Fits}

We performed 1000 geometry optimizations for the quark sector, systematically exploring coordinate assignments up to maximum coordinate value 5. Each geometry was optimized with 5 random seeds to ensure robustness, with optimization runs using 200 iterations per seed.

The best performing geometry achieves a CKM loss of 0.154872, with individual parameter matches:
\begin{itemize}
\item $V_{us}$: within 0.000113 of PDG target (0.22500) --- 0.05\% error
\item $V_{cb}$: within 0.000002 of PDG target (0.04182) --- 0.005\% error
\item $V_{ub}$: within 0.000000 of PDG target (0.00382) --- perfect match
\end{itemize}

While strict survivors (all parameters within experimental ranges simultaneously) are rare (0\% rate when using PDG experimental uncertainties), individual CKM parameters can be matched very precisely. This suggests the envelope-dominated regime can achieve high accuracy on individual observables, but simultaneously matching all observables within strict experimental ranges requires fine-tuning.

The best geometry uses coordinates $Q=(3,4)$, $U=(2,3,4)$, $D=(0,1,2)$, with optimal parameters:
\begin{itemize}
\item $\sigma = 4.72$
\item $k = 1.82$
\item $\alpha = 3.92$
\item $\eta = 1.51$
\item $\varepsilon_u = 0.010$
\item $\varepsilon_d = 0.010$
\end{itemize}

Notably, both $\varepsilon_u$ and $\varepsilon_d$ are at the lower bound of their allowed range, confirming that the envelope-dominated regime achieves hierarchy primarily through envelope suppression rather than phase interference.

Quark mass predictions for the best geometry include $m_c = 1.56$ GeV (target: 1.27 GeV), showing reasonable agreement with experimental values. The trade-off between CKM mixing accuracy and mass accuracy is captured by the Pareto frontier analysis.

\begin{figure}[htbp]
\centering
\includegraphics[width=0.8\textwidth]{figures/quark_pareto_ckm_mc.png}
\caption{Pareto frontier for quark sector showing trade-off between CKM loss and charm quark mass $m_c$. The boundary follows an exponential decay pattern $y = 15.587 \times \exp(-2.873x)$ with $R^2 = 0.8747$.}
\label{fig:quark_pareto}
\end{figure}

\subsection{Universal Pareto Envelope}

Analysis of the Pareto frontier reveals a universal exponential decay pattern in the trade-off between CKM loss and charm quark mass (Fig.~\ref{fig:quark_pareto}). The boundary is well-described by:
\begin{equation}
m_c = 15.587 \times \exp(-2.873 \times L_{\text{CKM}}),
\end{equation}
with goodness-of-fit $R^2 = 0.8747$. This exponential decay suggests a universal scaling behavior in the optimization landscape, where improvements in CKM mixing come at exponentially increasing cost in mass accuracy.

\subsection{Parameter Attribution}

Optimal parameters for quark sector geometries cluster around:
\begin{itemize}
\item $\sigma = 4.99 \pm 1.18$ (range: 0.59--6.00)
\item $k = 1.40 \pm 0.57$ (range: 0.10--2.00)
\item $\alpha = 2.98 \pm 1.58$ (range: 0.00--6.28)
\item $\eta = 2.95 \pm 1.47$ (range: 1.00--5.00)
\item $\varepsilon_u = 0.14 \pm 0.16$ (range: 0.01--0.50)
\item $\varepsilon_d = 0.20 \pm 0.19$ (range: 0.01--0.50)
\end{itemize}

The envelope width $\sigma \approx 5$ and phase parameter $\alpha \approx \pi$ emerge as natural scales for the quark sector.

\section{Results: Lepton Sectors}

\subsection{Charged Leptons: Phase-Sensitive Regime}

The phase-sensitive regime for charged leptons achieves exceptional performance, with a 60\% survivor rate (60 out of 100 geometries matching all experimental mass ranges). All survivors achieve perfect mass matches with errors $< 0.01\%$.

The best geometry achieves a total loss of $2.45 \times 10^{-11}$, with mass predictions:
\begin{itemize}
\item $m_e = 0.000511$ GeV (target: 0.000511 GeV) --- perfect match
\item $m_\mu = 0.105657$ GeV (target: 0.105658 GeV) --- 0.00\% error
\item $m_\tau = 1.776860$ GeV (target: 1.776860 GeV) --- perfect match
\end{itemize}

Z-scores for survivors are excellent: $Z(m_e) = 0.10$, $Z(m_\mu) = 0.27$, $Z(m_\tau) = 0.85$, all well below 1.0, indicating agreement within $1\sigma$.

\begin{figure}[htbp]
\centering
\includegraphics[width=0.8\textwidth]{figures/lepton_pareto_loss_me.png}
\caption{Pareto frontier for charged lepton sector showing total loss vs electron mass. The boundary follows a perfect exponential decay $y = 5.110 \times 10^{-4} \times \exp(-0.012x)$ with $R^2 = 1.0000$, indicating highly efficient optimization in the phase-sensitive regime.}
\label{fig:lepton_pareto}
\end{figure}

The Pareto boundary (Fig.~\ref{fig:lepton_pareto}) shows a perfect exponential fit:
\begin{equation}
m_e = 5.110 \times 10^{-4} \times \exp(-0.012 \times L_{\text{total}}),
\end{equation}
with $R^2 = 1.0000$. This remarkable fit suggests the phase-sensitive regime produces an extremely efficient optimization landscape, with phase parameter variation $(k_e, \eta_e)$ enabling precise resolution of the lepton mass hierarchy.

\subsection{Neutrinos: Metric-Dominated Regime}

The metric-dominated regime for neutrinos achieves a 45\% survivor rate (216 out of 480 optimizations). PMNS angle matches are excellent:
\begin{itemize}
\item $\theta_{12} = 0.569 \pm 0.023$ (target: 0.583) --- 2.40\% error
\item $\theta_{23} = 0.783 \pm 0.006$ (target: 0.785) --- 0.21\% error
\item $\theta_{13} = 0.1490 \pm 0.0008$ (target: 0.149) --- 0.02\% error
\end{itemize}

The best geometry achieves a PMNS loss of $1.36 \times 10^{-8}$, with angle predictions matching experimental values to within 0.02--2.40\%. Z-scores for survivors are excellent: $Z(\theta_{12}) = 0.62$, $Z(\theta_{23}) = 0.26$, $Z(\theta_{13}) = 0.05$, all well below 1.0.

\begin{figure}[htbp]
\centering
\includegraphics[width=0.8\textwidth]{figures/neutrino_pareto_pmns_genv.png}
\caption{Pareto frontier for neutrino sector showing PMNS loss vs envelope compression factor $g_{\text{env}}$. Colors indicate $g_{\text{env}}$ values. The optimal compression occurs at $g_{\text{env}} \approx 0.5$--0.6, where information loss under compression leads to emergent anarchy in PMNS mixing angles.}
\label{fig:neutrino_pareto}
\end{figure}

Optimal envelope compression occurs at $g_{\text{env}} \approx 0.5$--0.6 (Fig.~\ref{fig:neutrino_pareto}), where compression of the effective envelope width $\sigma_{\text{eff}} = \sigma \times g_{\text{env}}$ leads to information loss, resulting in emergent anarchy in PMNS mixing angles. This provides a natural explanation for why neutrino mixing angles are large (anarchy) while quark mixing angles are small (hierarchy): the neutrino sector probes a metric-dominated regime where compression washes out structure.

\subsection{Universal Parameter Commonalities}

Analysis across all three sectors reveals universal parameter values that transcend regime boundaries:

\textbf{Envelope Width $\sigma$:} All sectors converge to $\sigma \approx 4$--5:
\begin{itemize}
\item Quark: $4.99 \pm 1.18$
\item Charged Lepton: $4.40 \pm 1.39$
\item Neutrino: $4.94 \pm 1.39$
\end{itemize}

This universal envelope width suggests a common interference length scale across all fermion sectors, independent of the specific projection regime.

\textbf{Phase Parameter $\alpha$:} All sectors cluster around $\alpha \approx 2.5$--3.0 (near $\pi$):
\begin{itemize}
\item Quark: $2.98 \pm 1.58$
\item Charged Lepton: $2.58 \pm 1.55$
\item Neutrino: $2.46 \pm 1.04$
\end{itemize}

The clustering near $\pi$ suggests optimal phase interference occurs at this value, possibly corresponding to destructive interference patterns that create natural hierarchies.

\textbf{Envelope Suppression $\varepsilon$:} Shows systematic progression across regimes:
\begin{itemize}
\item Quark: $\varepsilon_u = 0.14$, $\varepsilon_d = 0.20$ (lowest)
\item Charged Lepton: $\varepsilon_e = 0.24$ (intermediate)
\item Neutrino: $\varepsilon_\nu = 0.44$, $\varepsilon_e = 0.36$ (highest)
\end{itemize}

This progression validates the three-regime framework: envelope-dominated (low $\varepsilon$) → phase-sensitive (medium $\varepsilon$) → metric-dominated (high $\varepsilon$).

\begin{figure}[htbp]
\centering
\includegraphics[width=0.8\textwidth]{figures/parameter_commonalities.png}
\caption{Parameter distribution comparisons showing universal values across sectors. Left: envelope width $\sigma$ distributions. Right: phase parameter $\alpha$ distributions. Both show remarkable consistency across quark, charged lepton, and neutrino sectors.}
\label{fig:param_common}
\end{figure}

\begin{figure}[htbp]
\centering
\includegraphics[width=0.8\textwidth]{figures/regime_comparison_survivors.png}
\caption{Survivor rates across the three projection regimes. Charged lepton sector (phase-sensitive) achieves 60\% survivors, neutrino sector (metric-dominated) achieves 45\% survivors, while quark sector (envelope-dominated) achieves 0\% strict survivors, demonstrating the varying difficulty across regimes.}
\label{fig:regime_comparison}
\end{figure}

\section{Discussion}

\subsection{Interpretation}

The universal interference kernel framework provides a coherent explanation for flavor patterns across all fermion sectors. The three projection regimes correspond to different ways of sampling the same underlying kernel structure:

\textbf{Envelope-Dominated Regime:} Quarks probe the baseline kernel where envelope suppression dominates. Phase interference provides subdominant corrections, allowing for hierarchical CKM mixing with small angles.

\textbf{Phase-Sensitive Regime:} Charged leptons probe variations in phase parameters $(k_e, \eta_e)$, enabling resolution of the muon mass hierarchy. The phase-sensitive nature allows for precise tuning of masses while maintaining the overall hierarchical structure.

\textbf{Metric-Dominated Regime:} Neutrinos probe compressed envelope widths via $g_{\text{env}} < 1$, leading to information loss and emergent anarchy. The compression washes out structure, explaining why neutrino mixing angles are large while quark mixing angles are small.

The universal parameter values ($\sigma \approx 4$--5, $\alpha \approx \pi$) suggest fundamental scales in the interference mechanism that are independent of the specific projection regime.

\subsection{Boundary Analysis}

Analysis of Pareto boundaries reveals universal exponential decay patterns:

\begin{figure}[htbp]
\centering
\includegraphics[width=0.8\textwidth]{figures/boundary_shape_comparison.png}
\caption{Normalized boundary comparison across all three sectors. All boundaries show exponential decay patterns when normalized to [0,1] in both dimensions, suggesting universal scaling behavior in the optimization landscape.}
\label{fig:boundary_comparison}
\end{figure}

\begin{figure}[htbp]
\centering
\includegraphics[width=0.8\textwidth]{figures/boundary_fits.png}
\caption{Boundary fits for all three sectors. Quark (left) and charged lepton (middle) boundaries show excellent exponential fits, while neutrino boundary (right) exhibits layered structure due to multiple $g_{\text{env}}$ values.}
\label{fig:boundary_fits}
\end{figure}

\begin{figure}[htbp]
\centering
\includegraphics[width=0.8\textwidth]{figures/boundary_curvature.png}
\caption{Curvature profiles along boundaries for all three sectors. All boundaries are smooth, convex frontiers with varying curvature. Charged lepton boundary shows the smoothest profile (consistent with perfect exponential fit).}
\label{fig:boundary_curvature}
\end{figure}

When normalized to [0,1] in both dimensions, all boundaries show similar exponential decay patterns (Fig.~\ref{fig:boundary_comparison}), suggesting a universal scaling behavior in the optimization landscape. The boundaries are smooth, convex frontiers (Fig.~\ref{fig:boundary_curvature}), indicating well-posed optimization problems with efficient frontiers.

### 6.3 Anarchy as Emergence

A key insight of this framework is that neutrino anarchy emerges from information loss under metric compression. The neutrino sector probes $g_{\text{env}} < 1$, compressing the effective envelope width and reducing information content. This compression washes out structure, leading to anarchy in PMNS mixing angles.

This provides a natural explanation for the observation that:
\begin{itemize}
\item Quark mixing angles are small (hierarchy) --- envelope-dominated regime preserves structure
\item Neutrino mixing angles are large (anarchy) --- metric-dominated regime loses structure
\end{itemize}

Both patterns emerge from the same universal kernel, just probed in different regimes. This unifies the understanding of quark and neutrino mixing within a single framework.

\subsection{Scope and Limitations}

This model operates at the effective field theory level, providing a structural framework rather than a fundamental theory. The model makes no claims about:
\begin{itemize}
\item UV completion or underlying microscopic theory
\item Why the universal kernel takes this particular form
\item How the discrete geometry arises from fundamental physics
\end{itemize}

The model successfully organizes Yukawa couplings within this framework, but the framework itself remains phenomenological. Future work may explore connections to UV physics, but such exploration is beyond the scope of this structural analysis.

\section{Conclusions and Further Study}

We have presented a unified framework for organizing Yukawa couplings through a universal interference kernel with three projection regimes. Key findings include:

\begin{enumerate}
\item \textbf{Three-Regime Framework:} Quarks, charged leptons, and neutrinos correspond to envelope-dominated, phase-sensitive, and metric-dominated regimes, respectively.

\item \textbf{Universal Parameters:} Envelope width $\sigma \approx 4$--5 and phase parameter $\alpha \approx \pi$ emerge universally across all sectors, suggesting fundamental scales in the interference mechanism.

\item \textbf{Survivor Rates:} Charged lepton sector achieves 60\% survivors, neutrino sector achieves 45\% survivors, with all survivor Z-scores $< 1.0$, demonstrating excellent statistical agreement with experimental data.

\item \textbf{Exponential Boundaries:} All Pareto boundaries exhibit exponential decay patterns, suggesting universal scaling behavior in the optimization landscape.

\item \textbf{Anarchy as Emergence:} Neutrino anarchy emerges from information loss under metric compression, providing a natural explanation for large neutrino mixing angles within the same framework that produces small quark mixing angles.
\end{enumerate}

\subsection{Further Study}

Several directions for future work emerge naturally from this framework:

\textbf{Geometry Space Extensions:} Current optimizations explore geometries up to coordinate value 5 for quarks. Extending to higher coordinate values may reveal additional geometries that achieve simultaneous matching of all observables. However, computational cost scales as the number of geometries increases.

\textbf{CP Violation:} While this work focuses on mixing angles and masses, the framework naturally includes CP-violating phases through the complex nature of Yukawa matrices. Future work could explore CP violation predictions, comparing to experimental constraints on the CKM phase $\delta_{\text{CKM}}$ and PMNS phase $\delta_{\text{PMNS}}$.

\textbf{UV Physics Connections:} The universal kernel form may emerge from UV physics models, such as string compactifications, extra-dimensional models, or flavor symmetries. Exploring connections to UV completions could provide deeper insight into the origin of the kernel structure.

\textbf{RGE Evolution:} Yukawa couplings evolve with energy scale according to renormalization group equations (RGEs). The kernel parameters determined at the weak scale may need to be evolved to unification scales or high scales where UV physics operates. Understanding RGE effects could constrain or guide UV completions.

\textbf{Statistical Validation:} While parameter commonalities ($\sigma \approx 4$--5, $\alpha \approx \pi$) are suggestive, rigorous statistical validation could confirm whether these values represent universal scales or coincidences. Kolmogorov-Smirnov tests and other statistical methods could quantify the significance of these commonalities.

\textbf{Survivor Analysis:} The varying survivor rates across regimes (60\% for leptons, 45\% for neutrinos, 0\% for quarks) suggest different optimization landscapes. Deeper analysis of why certain regimes produce more survivors could reveal insights into the structure of the optimization space.

\textbf{Boundary Mathematics:} The exponential decay patterns in Pareto boundaries suggest universal scaling laws. Deriving these patterns analytically, rather than empirically, could provide deeper understanding of the optimization landscape structure.

The framework provides a coherent explanation for flavor patterns across all fermion sectors, unifying quark and neutrino mixing within a single universal kernel structure. The three-regime framework offers a natural organization of fermion flavor physics, with each regime probing different aspects of the underlying universal structure.

\section*{Acknowledgments}

The author thanks the PDG collaboration for maintaining comprehensive experimental data compilations.

\bibliographystyle{apsrev4-1}
\bibliography{references}

\end{document}
